\documentclass[a4paper,12pt]{article}
\usepackage[utf8]{inputenc}
\usepackage[T1]{fontenc}
\usepackage[lmargin=2cm,rmargin=2cm]{geometry}
\usepackage{graphicx}
\usepackage{float}
\usepackage{longtable}
\title{System wspomagający pracę ratowników górskich Babiej Góry}
\author{Arkadiusz Błasiak, Szymon Nowak}
\date{}
\begin{document}
\maketitle
\newpage
\section{Cel}
System ma na celu wspomagać pracę ratowników górskich Babiej Góry. Dokładne informacje o warunkach pogodowych i lokalizacje pobierane z BTS’ów, pomogą ratownikom monitorować szlaki z większą dokładnością, przyspieszyć czas reakcji oraz podjąć decyzję o rozpoczęciu akcji ratunkowej.
\section{Podstawowe założenia}
\begin{enumerate}
\item Babia Góra podzielona jest na 11 szlaków o różnej długości i różnym poziomie trudności (żółty, zielony, niebieski, czarny, czerwony). Dla każdego szlaku obliczana będzie skala zagrożenia
\item Na skalę zagrożenia każdego szlaku wpływają:
\begin{itemize}
\item warunki atmosferyczne
\begin{itemize}
\item wiatr
\item deszcz
\item mgła
\item temperatura
\end{itemize}
\item zagrożenie lawinowe
\item poziom trudności szlaku
\end{itemize}
\item Informacje o warunkach atmosferycznych pobierane są z czujników rozmieszczonych w różnych miejscach na szlakach Babiej Góry. Informacje o lokalizacji komórek turystów pobierane są BTS’ów
\item Każdy z powyższych czynników posiada osobną skalę. Końcowy poziom stanu alarmowego dla każdego szlaku oblicza się ze wzoru: (suma skal dla warunków atmosferycznych + zagrożenie lawinowe + poziom trudności szlaku)/4.
\item W zależności od uzyskanego wyniku system sugeruje określone akcje ratowników
\end{enumerate}
\newpage
\section{Szlaki babiej góry}
\begin{center}
    \begin{longtable}{ | l | l | p{5cm} | l | l | p{2cm} |}
    \hline
    \textbf{Lp} & \textbf{Kolor} & \textbf{Trasa} & \textbf{Czas przejścia} & \textbf{Długość trasy} & \textbf{Poziom trudności}\\ \hline
    1 & żółty & Markowe Szczawiny -- Sucha Kotlinka -- Diablak & 1h 30min & 3km & 1\\\hline
    2 & żółty & Zawoja Czatoża -- Fickowe Rozstaje -- Górny Płaj -- Markowe
Szczawiny & 3h 20min & 5km & 1\\\hline
	3 & niebieski & Zawoja Czatoża -- Markowe Rówienki -- Zawoja Markowa --Ryzowana -- Zawoja Policzne & 3h 30min & 7,50km & 2\\\hline
	4 & niebieski & Zawoja Policzne -- Polana rowiarki -- Markowe Szczawiny & 3h & 11km & 2\\\hline
	5 & zielony & Zawoja Markowa -- Pośredni Bór -- Markowe Szczawiny & 1h 20min &  & 3\\\hline
	6 & zielony & Górny Płaj -- Sokolica & 45min & 1,5km & 3\\\hline
	7 & zielony & Przełęcz Jałowiecka -- Mała Babia Góra -- Przełęcz Brona & 2h & 4km & 3\\\hline
	8 & zielony & Przywarówka -- Głodna Woda -- Diablak & 2h 30min & 2,3km & 3\\\hline
	9 & zielony & Polana Krowiarki -- Hala Śmietanowa -- Zubrzyca Górna & 2h & 3km & 3\\\hline
	10 & czerwony & Polana Krowiarki -- Sokolica -- Kępa -- Gówniak -- Diablak -- Przełęcz Brona -- Markowe Szczawiny -- Fickowe Rozstaje -- Przełęcz Jałowiecka & 6h & 14,5km & 4\\\hline
	11 & czarny & Podryzowana -- Ryzowana -- Markowe Szczawiny & 2h 30min & 3,5km & 4\\\hline
    \end{longtable}
\end{center}
\newpage
\section{Czynniki atmosferyczne}
\begin{center}
    \begin{longtable}{ | l | p{5cm} | l | l |}
    \hline
    \textbf{Czynnik} & \textbf{Opis} & \textbf{Skala} & \textbf{Wytyczne} \\ \hline
    wiatr & Przy wietrze o prędkości 80–100
km/godz trudno utrzymać się na nogach
pokonywanie wysokogórskich szlaków
jest często niemożliwe, grozi
przewróceniem lub wręcz
zdmuchnięciem w przepaść. Poruszając
się w terenach leśnych, narażamy się na
przygniecenie przez łamane wiatrem i
wywracające się drzewa lub uderzenie
spadającymi gałęziami. &
1--3 & \parbox[t]{5cm}{1: do 30km/h\\
2: 30-80km/h\\
3: powyżej 80km/h} \\ \hline
	mgła&Jej obecność może nie tylko
uniemożliwić oglądanie widoków, ale przede wszystkim znacznie utrudnić
orientację w terenie. Szczególnie
niebezpieczna jest mgła w warunkach
zimowych, gdy biel śniegu całkowicie
zlewa się z bielą mgły. Nie widząc
punktów odniesienia, nie będziemy w
stanie określić położenia w terenie,
kierunku marszu, a nawet ocenić
stromizny stoku, na którym się
znajdujemy.& 1--3 & \parbox[t]{5cm}{1: zauważalna
mgła, która nie wpływa jednak na
poruszanie się po
szlaku\\
2: słaba widoczność\\
3: znikoma
widoczność,
znaczące
utrudnienia w
orientacji
terenowej} \\ \hline
 	temperatura & Na szczytach temperatura z reguły jest
niższa niż u stóp gór. Każde 100 metrów
wyżej, to około 1 st. C mniej. Uczucie
zimna jest dodatkowo potęgowane przez
wiatr. Słaby wiaterek przy termometrze
wskazującym zero stopni spowoduje
odczuwanie zimna równe około 3 st. 
C,
podmuch (5 m/s) – prawie 9 st. 
C, wiatr 15
m/s aż 18 st.
C. Ten sam wiatr przy
umiarkowanym mrozie 10 st.
C da
odczucie poniżej 33 st.
C. & 1--3 & \parbox[t]{5cm}{1: powyżej 15 lub
poniżej 5\\
2: powyżej 24 lub
poniżej -3\\
3: powyżej 30 lub
poniżej -15}\\ \hline
	deszcz, burza & W górach jest niebezpieczne podczas
burzy ze względu na ilość dni burzowych
w ciągu roku, wysokość brak
możliwości schronienia się. Dodatkowym
zagrożeniem są barierki, łańcuchy,
drabiny, brak dobrze przewodzacych
warstw. Burza jest niebezpieczna, jeżeli
pomiędzy błyskiem a grzmotem jest
mniej niz 515
sekund. Znaczy to, że
piorun uderzył bliżej niż 25
km od nas.
Dodatkowo opady deszczu podwyższają
trudności tras związane z poruszaniem
sie po nich. & 1--3 & \parbox[t]{5cm}{1:
deszcz - od 15 do
40mm
wiatr - od 0 do
30km/h\\
2:
deszcz - od 41 do
70mm
wiatr - od 31 do
80kmh\\
3:
deszcz - powyżej
70mm
wiatr - powyżej
80kmh}\\ \hline
    \end{longtable}
\end{center}
\newpage
\section{Zagrożenie lawinowe}
\begin{center}
    \begin{longtable}{ | l | p{10cm} |}
    \hline
    \textbf{Stopień zagrożenia} & \textbf{Prawdopodobieństwo zejścia lawiny}\\ \hline
    5 & Istnieje prawdopodobieństwo samoczynnego schodzenia wielu dużych, niejednokrotnie
również bardzo dużych lawin, także w terenie
umiarkowanie stromym. \\\hline
	4 & Wyzwolenie lawiny jest prawdopodobne na
licznych stromych stokach już przy małym
obciążeniu dodatkowym. W niektórych
przypadkach możliwe jest samorzutne
schodzenie licznych średnich, a często
również dużych lawin.\\\hline
	3 & Wyzwolenie lawiny jest możliwe już przy
małym obciążeniu dodatkowym wszystkim na
wskazanych stromych stokach. W niektórych
przypadkach możliwe jest samorzutne
schodzenie średnich, a sporadycznie także
dużych lawin.\\\hline
	2 & Wyzwolenie lawiny jest możliwe zwłaszcza
przy dużym obciążeniu dodatkowym przede
wszystkim na wskazanych stromych stokach.
Nie należy spodziewać się samorzutnego
schodzenia dużych lawin.\\\hline
	1 & Wyzwolenie lawiny na ogół jest możliwe tylko
przy dużym obciążeniu dodatkowym w
nielicznych miejscach w terenie ekstremalnie
stromym. Możliwe jest samorzutne schodzenie
lawin głównie w postaci zsuwów i małych
lawin. \\\hline
    \end{longtable}
\end{center}
\newpage
\section{Sytuacje alarmowe -- turyści}
\begin{center}
\begin{longtable}{ | p{15cm} | }
	\textbf{Sytuacja alarmowa}\\\hline
	Turysta przebywa w jednym miejscu przez
dłuższy czas. Możliwe zasłabnięcie, stracenie
przytomności.\\\hline
\end{longtable}
\end{center}
\newpage
\section{Stany alarmowe dla szlaków}
\end{document}